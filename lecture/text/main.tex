\section{Text}

\subsection{Special Characters}

\begin{frame}
    \frametitle{Special Characters}
    Some special symbols can't be directly used since they are reserved by \LaTeX:
    \begin{center}
        \begin{tabular}{llllll}
            \LC|\#|               & \#                                                                                                                            \\
            \samplesymbol{\#}{\#} & \samplesymbol{\$}{\$} & \samplesymbol{\%}{\%} & \samplesymbol{\&}{\&} & \samplesymbol{\~{}}{\~{}} & \samplesymbol{\`{}}{\`{}} \\
            \samplesymbol{\{}{\{} & \samplesymbol{\}}{\}} & \samplesymbol{\_}{\_} &
            \multicolumn{2}{l}{\samplesymbol{backslash}{$\backslash$}}
        \end{tabular}
    \end{center}

    Many \LaTeX\ starters are confused with how to correctly print quotes, hyphens and dots.\\
    \pause
    \`{} prints a left single quote, ' prints a right single quote.\\
    \pause
    \`{}\`{} prints a left double quote, '' prints a right double quote.\\
    \pause
    one hyphen (-) print like - \\
    \pause
    two hyphens ({-}{-}) print like -- \\
    \pause
    three hyphens ({-}{-}{-}) print like ---\\
    \pause
    \samplecommand{dots} prints the dots with a correct format (\dots) instead of directly use three dots (...)
\end{frame}

\begin{frame}{Warning}
    \begin{warning}
        Do NOT use {\tt "} and {\tt '} to print quotes.
    \end{warning}
\end{frame}

\begin{frame}
    \frametitle{Deal with unfamiliar symbols}
    Sometimes you may want to deal with symbols you have never seen. In this case, you may refer to \url{http://detexify.kirelabs.org/classify.html} to find out how to output the character.
\end{frame}

\subsection{Fonts}

\begin{frame}
    \frametitle{Basic commands about fonts}
    First, lets start with some commands that transform font types
    \pause
    \begin{itemize}
        \item \sampletext{bf}{\bf}
        \item \sampletext{it}{\it}
        \item \sampletext{rm}{\rm}
        \item \sampletext{sc}{\sc}
        \item \sampletext{sf}{\sf}
        \item \sampletext{sl}{\sl}
        \item \sampletext{tt}{\tt}
    \end{itemize}
    \pause
    Note that the commands that transform font types influence the text in the whole scope (\structure{\{...\}}) until another font type is specified. For example, how to use the first command \samplecommand{bf} is shown below\\[0.5em]
    \{\samplecommand{bf} Sample Text\}
\end{frame}

\begin{frame}[fragile]
    Sometimes we don't want to transform  all the font types, instead, we can only change the font type of some specified text.
    \pause
    \begin{example}
        \begin{minted}{latex}
\textbf{Sample text}
		\end{minted}
    \end{example}
    \pause
    There are more options for fonts.
    \begin{itemize}
        \item \sampletext{textit}{\textit}
        \item \sampletext{textsc}{\textsc}
        \item \sampletext{texttt}{\texttt}
    \end{itemize}
    \pause
    However, in a math environment (will be introduced later), some other commands should be used
    \begin{itemize}
        \item \samplecommand{mathbf} - $\mathbf{Sample\ Text}$
        \item \samplecommand{mathsf} - $\mathsf{Sample\ Text}$
    \end{itemize}
    \pause
    Note that the math environment doesn't include all of the font types on the previous page. More information about font types can be found \href{http://www.cnblogs.com/make217/p/6123532.html}{here}.
\end{frame}

\begin{frame}
    Font size can also be easily modified
    \begin{itemize}
        \item \sampletext{tiny}{\tiny}
        \item \sampletext{scriptsize}{\scriptsize}
        \item \sampletext{footnotesize}{\footnotesize}
        \item \sampletext{small}{\small}
        \item \sampletext{normalsize}{\normalsize}
        \item \sampletext{large}{\large}
        \item \sampletext{Large}{\Large}
        \item \sampletext{LARGE}{\LARGE}
        \item \sampletext{huge}{\huge}
        \item \sampletext{Huge}{\Huge}
    \end{itemize}
\end{frame}

\begin{frame}[fragile]
    \frametitle{Build a colorful document}
    Changing the color is similar to changing font types. \medskip

    If you want to transform to a color (like transforming to bold with \LC{\bf}), you can use \LC{\color{name}}. \smallskip

    Similarly, you can use \LC{\textcolor{name}} like \LC{\textbf}.\smallskip

    The background color of the whole page can be set using \LC{\pagecolor{name}}.\medskip

    \pause

    There are some defined color \structure{name} in the \structure{xcolor} package.\medskip

    \begin{tabular}{lllll}
        \samplecolorbox{black}  & \samplecolorbox{gray}     & \samplecolorbox{olive}   & \samplecolorbox{teal}  & \samplecolorbox{blue}      \\
        \samplecolorbox{green}  & \samplecolorbox{orange}   & \samplecolorbox{violet}  & \samplecolorbox{brown} & \samplecolorbox{lightgray} \\
        \samplecolorbox{pink}   & \samplecolorbox{white}    & \samplecolorbox{cyan}    & \samplecolorbox{lime}  & \samplecolorbox{purple}    \\
        \samplecolorbox{yellow} & \samplecolorbox{darkgray} & \samplecolorbox{magenta} & \samplecolorbox{red}                                \\
    \end{tabular}
    \medskip

    You can find more information in the documentation of \structure{xcolor} (\alert{texdoc} \structure{xcolor})
\end{frame}

\subsection{Underline}

\begin{frame}[fragile]
    \frametitle{Ulem package}
    If you want to add some lines on the text, use the \structure{ulem} package.
    \begin{command}
        \begin{minted}{latex}
\usepackage{ulem}
\uline{Sample Text}
		\end{minted}
    \end{command}
    \pause
    There are different kinds of lines supported:
    \begin{itemize}
        \item \sampletext{uline}{\uline}
        \item \sampletext{uuline}{\uuline}
        \item \sampletext{uwave}{\uwave}
        \item \sampletext{sout}{\sout}
        \item \sampletext{xout}{\xout}
        \item \sampletext{dashuline}{\dashuline}
        \item \sampletext{dotuline}{\dotuline}
    \end{itemize}
\end{frame}

\subsection{Enumeration}
\begin{frame}[fragile]
    \frametitle{Enumerate}
    When you need to enumerate some items as a list, you may use the \structure{enumerate} package.
    \begin{command}
        \begin{minted}{latex}
\usepackage{enumerate}
\begin{enumerate}[style]
\item % ...
\item % ...
\item % ...
\end{enumerate}
		\end{minted}
    \end{command}
    \pause
    This will generate a normal list with the serial numbers in the specified \structure{style}, which could be the following (as example)
    \begin{itemize}
        \item \alert{1} - 1, 2, 3, 4, ...
        \item \alert{(i)} - (i), (ii), (iii), (iv), ...
        \item \alert{[1.]} - [1.], [2.], [3.], [4.], ...
    \end{itemize}
\end{frame}

\begin{frame}[fragile]
    \frametitle{Itemize}
    If you want to generate an unordered list, use \structure{itemize} instead of \structure{enumerate}.
    \begin{command}
        \begin{minted}{latex}
\begin{itemize}
\item[style] % ...
\item[style] % ...
\item[style] % ...
\end{itemize}
		\end{minted}
    \end{command}
    \pause
    In this case, \structure{style} must be added after each item, which is different from that in \structure{enumerate}, and the symbol displayed in the beginning of each item will be exactly same as the \structure{style}. If \structure{style} is not added, a default style will be used.
\end{frame}

\subsection{Alignment}

\begin{frame}[fragile]
    \frametitle{Alignment}
    If you want to align a paragraph of text, use these three environments for left/center/right align.
    \begin{command}
        \begin{minted}{latex}
\begin{flushleft/center/flushright}
% ...
\end{flushleft/center/flushright}
			\end{minted}
    \end{command}
    \pause
    However, if only a single line needs to be aligned, use these three commands.
    \begin{command}
        \begin{minted}{latex}
\leftline{text}
\centerline{text}
\rightline{text}
		\end{minted}
    \end{command}
\end{frame}

\subsection{Spaces, lines and pages}

\begin{frame}
    \frametitle{Spaces may be confusing}
    There are defined command of spaces in different width and usages.
    \begin{itemize}
        \item \textvisiblespace\ - the basic space in \LaTeX\. Note that any number of spaces or tabs is equal to one space, and the space after a command is ignored. If you want to add an extra space, use \alert{\textbackslash\textvisiblespace\ } which makes a 1/3\,em space (1 em is approximately the width of an \structure{M} in the current font)
        \item \~{} - If two words can't be separated on two lines, you can tell \LaTeX\ about it using a tie (\~{}), such as Prof.\~{}Hamade (Prof.~Hamade).
        \item  \samplecommand{,} - makes a 1/6\,em space, commonly used before units (notice the space before em on this page)
        \item  \samplecommand{;} - makes a 2/7\,em space
        \item  \samplecommand{quad} - makes a 1\,em space
        \item  \samplecommand{qquad} - makes a 2\,em space
        \item  \samplecommand{phantom}\{\structure{text}\} - makes actually the space of \structure{text}, but \structure{text} will be invisible.
    \end{itemize}
\end{frame}

\begin{frame}
    \frametitle{Separate contents into lines and pages}
    Here are some basic commands about lines and pages in \LaTeX,  you will use them everywhere.
    \pause
    \begin{itemize}
        \item \samplecommand{newline} - begin a new line
        \item \alert{\textbackslash\textbackslash} - begin a new line (not recommended\footnotemark[1])
        \item \samplecommand{par} - begin a new paragraph (a new line with indent)
        \item \samplecommand{newpage} - begin a new page
        \item \alert{\%} - begin a line comment
    \end{itemize}
    \footnotetext[1]{According to Manuel Charlemagne, \alert{\textbackslash\textbackslash} should only be used for a force break (where \samplecommand{newline} doesn't work).}
\end{frame}

\begin{frame}{Warning}
    \begin{warning}
        Never use \alert{\textbackslash\textbackslash} to create a new line in the text, it will lead to unexpected side effects.
    \end{warning}
  \end{frame}

\begin{frame}[fragile]
    \frametitle{Spacing}
    When trying to separate two paragraphs by a certain space, many new learners of \LaTeX\ may use multiple empty lines and linebreaks, which is a very dirty fix and is not so accurate. Actually, \LaTeX\ provides a precise spacing mechanism.
    \begin{command}
        \LC{\vspace{space}}\\
        \LC{\vspace*{space}}
    \end{command}
    \pause
    When trying to show the next paragraph or sentence precisely at the bottom of the current page, we can use
    \begin{command}
        \LC{\vfill}
    \end{command}
    between the contents of two paragraphs to separate them.
\end{frame}

\begin{frame}[fragile]
    \frametitle{Predefined skipping}

    More often\footnotemark[1], we don't need to think about the skipping space, we can use the predefined skipping commands to achieve a small, medium or big skip. They are actually particular cases of \LC{\vspace}

    \begin{command}
        \LC{\smallskip}\smallskip

        \LC{\medskip}\medskip

        \LC{\bigskip}\bigskip
    \end{command}

    \pause
    You may note that the effects are these skipping commands have been already shown above.
    \footnotetext[1]{According to Manuel Charlemagne, you should always use these skipping commands if possible instead of using \alert{\textbackslash\textbackslash} (as in many online tutorials).}
\end{frame}

\begin{frame}
    \frametitle{Spacing units}
    The \structure{space} can be anything representing a size, such as \structure{1cm}, \structure{2em} and \structure{10pt}. In \LaTeX, spacing units can be
    \begin{itemize}
        \item \structure{cm}
        \item \structure{mm}
        \item \structure{in} - inch, 1 inch = 2.54 cm
        \item \structure{pt} - 72 pt = 1 inch, the smallest unit in \LaTeX
        \item \structure{em} - 1em equals to the width of letter M
        \item \structure{ex} - 1ex equals to the width of letter x
        \item \LC{\linewidth} - the width of current line in the container
        \item \LC{\pagewidth} - the width of the page
        \item \LC{\pageheight} - the height of the page
        \item \LC{\textwidth} - the normal width of text on the page
        \item \LC{\textheight} - the normal height of text on the page
    \end{itemize}
\end{frame}

\subsection{Minipage and Multicolumn}

\begin{frame}[fragile]
    \frametitle{Minipage}
    \structure{minipage} is a very useful environment for dividing pages into a grid.
    \begin{example}
        \begin{multicols}{2}
            \inputminted{latex}{../text/minipage.tex}
        \end{multicols}
    \end{example}
\end{frame}

\begin{frame}
    The code above generate six minipages in a grid of 3 columns $\times$ 2 rows. Don't try to add up the width of minipages in a line for more than about \LC{0.98\linewidth} (since a minipage have a small margin on each side), or the last minipage may be on a new line. \\[0.5em]
    For each minipage, it can be seem as an independent \LaTeX\ document, where text, formulas, graphics, tables and etc. can be inserted, and most importantly, they won't affect each other. What's more, you can even use minipages in a minipage to form a multi-level nesting. \\
\end{frame}

\begin{frame}{Example}
    \begin{minipage}{0.32\linewidth}
  Part A
\end{minipage}
\hfill % Fill horizontal space
\begin{minipage}{0.32\linewidth}
  Part B
\end{minipage}
\hfill % Fill horizontal space
\begin{minipage}{0.32\linewidth}
  Part C
\end{minipage}
\vfill % Fill vertical space
\begin{minipage}{0.32\linewidth}
  Part D
\end{minipage}
\hfill % Fill horizontal space
\begin{minipage}{0.32\linewidth}
  Part E
\end{minipage}
\hfill % Fill horizontal space
\begin{minipage}{0.32\linewidth}
  Part F
\end{minipage}

\end{frame}

\begin{frame}[fragile]
    \frametitle{The multicol package}
    When typesetting contents with small line width and many lines (for example, source code), the \structure{multicol} package is recommended.
    \begin{command}
        \begin{minted}{latex}
\usepackage{multicol}
\begin{multicols}{cols}
    % contents on column one
    \breakcolumn % break the current column here
    % contents on column two
\end{multicols}
		\end{minted}
    \end{command}
    Here \structure{cols} is the number of columns, it must be specified. If \LC{\breakcolumn} is not used, the \structure{multicol} package will automatically balance the length of each column.
\end{frame}
